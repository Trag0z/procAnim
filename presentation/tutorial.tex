%Praesentationsmodus
\documentclass[t,aspectratio=169,dvipsnames]{beamer}
%Die Beameroption aspectratio legt das verwendete Seitenverhaeltnis fest
%aspectratio=169	16:9 Seitenverhaeltnis
%aspectratio=1610	16:10 Seitenverhaeltnis
%aspectratio=43		4:3 Seitenverhaeltnis
%Die Beameroption envcountsect nummeriert Umgebungen wie theorem pro section durch.
%Die Beameroption divpsnames wird an das xcolor Paket durchgereicht.

%Handout-Generierung mit Foliennotizen (statt obiger Zeile für den Präsentationsmodus verwenden)
%\documentclass[t,handout,aspectratio=169]{beamer}
%Foliennotizen
%\setbeameroption{show notes}

\usepackage[utf8]{inputenc}

% Deutsch
\usepackage[ngerman]{babel} 
\usepackage{bibgerm}

% Englisch
%\usepackage[english]{babel}

\input{templatesetup}

\usepackage{tikz}
\usepackage{lipsum}
\usepackage{listings}
\lstset
{
	basicstyle=\ttfamily, 
	keywordstyle=\color{blue}\bfseries\ttfamily,
	identifierstyle=\ttfamily, 
	stringstyle=\ttfamily,
	commentstyle=\color{ForestGreen},
	showstringspaces=false,
	framexleftmargin=7mm, 
	breaklines=true,
	tabsize=3,
	showtabs=false,
	frame=single, 
	rulesepcolor=\color{blue},
	numbers=left,
	linewidth=146mm,
	xleftmargin=8mm,
	language={C++},
}


% Stil des Literaturverzeichnisses
\bibliographystyle{geralpha}
%\bibliographystyle{alpha}
%\bibliographystyle{abstract}

%Bitte ausfuellen:
\title{Präsentationen mit Beamer}
\subtitle{Eine kurze Einführung mit Beispielen}
\author{Stefan Bodenschatz}
\institute{Hochschule Trier}
\date{\today}
\subject{Kurzeinführung Beamer}

%Inhaltsverzeichnisses bis auf subsubsection-Ebene:
%\setcounter{tocdepth}{3}

%Aktivieren, um am Anfang jeder Section ein Inhaltsverzeichnis zur Section anzuzeigen
%\AtBeginSection[]
%{
%\begin{frame}<beamer>
%\frametitle{Agenda}
%\tableofcontents[currentsection,hideothersubsections,sectionstyle=show/hide,subsubsectionstyle=show/show]
%\end{frame}
%}

%Aktivieren, um alles Schritt-fuer-Schritt einzublenden
%\beamerdefaultoverlayspecification{<+->}

\begin{document}

\begin{frame}
\titlepage
\end{frame}

\begin{frame}
\frametitle{Agenda}
\tableofcontents
%\tableofcontents[hideallsubsections] % Subsections ausblenden
%\tableofcontents[pausesections] %Sections Schritt-fuer-Schritt einblenden
\end{frame}

\section{Grundlagen}
\begin{frame}{Was ist Beamer?}{Eine Übersicht}
Das Beamer Paket für \LaTeX{} ermöglicht es, Präsentationsfolien zu erstellen und unterstützt dabei Features wie Animationen, die manuell nur mit viel Aufwand umgesetzt werden können.
Für die Details zu den Features siehe den Beamer User Guide\cite{BeamerDoc}.
\end{frame}

\begin{frame}
Folien in Beamer werden durch frame-Umgebungen definiert.
\end{frame}

\begin{frame}{Festlegen des Titels}{... und des Subtitels}
Die Frames verfügen über einen Titel und einen Subtitel.
Diese können entweder beim Öffnen der frame-Umgebung angegeben werden.
\end{frame}

\begin{frame}
\frametitle{Festlegen des Titels}
\framesubtitle{... und des Subtitels}
Oder sie werden durch entsprechende Befehle angegeben.
\end{frame}

\begin{frame}{Strukturierung}{Aufzählungen}
\begin{itemize}
	\item Fließtext ist meistens nicht sinnvoll für Präsentationen
	\item Aufzählungen sind oft besser geeignet
	\begin{itemize}
		\item Geben Struktur
		\item Sorgen für Übersicht
	\end{itemize}
\end{itemize}
\begin{enumerate}
	\item Auf nummerierte Aufzählungen können verwendet werden
	\item Wie hier zu sehen ist
	\begin{enumerate}
		\item Auch mit
		\item Unterpunkten
	\end{enumerate} 
\end{enumerate}
\end{frame}

\section{Strukturelemente}
\begin{frame}{alert-Text}
	Wichtige Teile im Text können \alert<2>{durch alert hervorgehoben} werden.
	\begin{itemize}[<+-|alert@+>]
		\item Schritt 1
		\item Schritt 2
		\item Schritt 3
	\end{itemize}
\end{frame}

\subsection{Environments}
\begin{frame}{Environments}
\begin{theorem}
	Umgebungen wie theorem können auch in Beamer genutzt werden.
\end{theorem}
\begin{proof}
	Proof wird auch unterstützt.
\end{proof}
\begin{figure}
	\includegraphics[width=3.5cm]{HochschuleLogo}
	\caption{Testbild in einem figure float}
\end{figure}
\begin{center}
	\includegraphics[width=3.5cm]{HochschuleLogo}\\
	\emph{Testbild in center-Umgebung}
\end{center}
\end{frame}

\begin{frame}{Mathe-Modus}
	Der Mathe-Modus kann wie in LaTeX üblich benutzt werden:\\ 
	$\mathcal{F}: x = y + \frac{\mathsf{z}}{3}, y \in \mathbb{N}, \mathsf{z} \in \mathfrak{B}$ 
\end{frame}

\subsection{Blocks}
\begin{frame}{Blocks}
\begin{block}{Dies ist ein Block}
	Blocks können zur Strukturierung des Frame-Inhalts genutzt werden.
\end{block}
\begin{exampleblock}{Dies ist ein Beispielblock}
	Inhalt...
\end{exampleblock}
\begin{alertblock}{Dies ist ein Alert-Block}
	Inhalt...
\end{alertblock}
\end{frame}

\section{Frameoptionen}
\begin{frame}[allowframebreaks]{allowframebreaks}
	Die Option allowframebreaks erlaubt es frames mit zu viel Inhalt umzubrechen. Dies ist besonders bei generiertem Inhalt wie dem Quellenverzeichnis sinnvoll.
	\\[1cm]
	\lipsum[1-2]
\end{frame}

\begin{frame}[fragile]{fragile}
\begin{lstlisting}
//Quellcode-Listings
//und andere Verbatim-Umgebungen
//setzen die fragile-Option des Frames voraus.
#include <iostream>
void main(){
	std::cout<<"Hallo Welt"<<std::endl;
}
\end{lstlisting}
\end{frame}

\section{Animationen}

\begin{frame}{Animieren mit Pause}
\begin{itemize}
	\item Schritt 1
	\item Schritt 2
	\pause
	\item Schritt 3
	\item Schritt 4
\end{itemize}
\end{frame}

\begin{frame}[<+->]{Ganzen Frame automatisch animieren}
	 %Das + steht immer fuer den naechsten Animationsschritt
\begin{itemize}
	\item Schritt 1
	\item Schritt 2
	\item Schritt 3
	\item Schritt 4
\end{itemize}
\end{frame}

\begin{frame}{Einzelne Umgebung automatisch animieren}
\begin{itemize}[<+->]
	\item Schritt 1
	\item Schritt 2
	\item Schritt 3
	\item Schritt 4
\end{itemize}
\end{frame}

\begin{frame}{Abweichende Animationsreihenfolge}{selbstdefiniert}
\begin{itemize}
	\item<1,3> Schritt 1
	\item<2-> Schritt 2
	\item<3-4> Schritt 3
	\item<5> Schritt 4
\end{itemize}
\end{frame}

\begin{frame}{Block-übergreifende Animation}{Übereinander}
\begin{block}{Bereich 1}
	\begin{itemize}
		\item<1-> Schritt 1
		\item<3-> Schritt 2
		\item<5-> Schritt 3
	\end{itemize}
\end{block}
\begin{block}{Bereich 2}
	\begin{itemize}
		\item<2-> Schritt 1
		\item<4-> Schritt 2
		\item<6-> Schritt 3
	\end{itemize}
\end{block}
\end{frame}
\note{Foliennotizen werden mit dem note-Befehl hinzugefügt.}

\begin{frame}{Block-übergreifende Animation}{Nebeneinander}
\begin{columns}[T]
\begin{column}[T]{0.5\textwidth}
\begin{block}{Bereich 1}
	\begin{itemize}
		\item<1-> Schritt 1
		\item<3-> Schritt 2
		\item<5-> Schritt 3
		\item<6-> Schritt 4
	\end{itemize}
\end{block}
\end{column}
\begin{column}[T]{0.5\textwidth}
\begin{block}{Bereich 2}
	\begin{itemize}
		\item<2-> Schritt 1
		\item<4-> Schritt 2
		\item<7-> Schritt 3
		\item<8-> Schritt 4
	\end{itemize}
\end{block}
\end{column}
\end{columns}
\end{frame}
\note{Diese Notizen werden nur ausgegeben, wenn die Option show notes gesetzt ist.}

\begin{frame}{Beliebige Inhalte ein- und ausblenden}
	\begin{itemize}[<+->]
		\item Schritt 1
		\item Schritt 2
		\item Schritt 3
		\item Schritt 4
	\end{itemize}
	\only<2>{Dieser Hinweis wird nur für Schritt 2 angezeigt.}
	\only<3>{Dieser Hinweis wird nur für Schritt 3 angezeigt.}
	Nachfolgender Text kann sich aber verschieben.
\end{frame}

\begin{frame}{Beliebige Inhalte ein- und ausblenden}
	\begin{itemize}[<+->]
		\item Schritt 1
		\item Schritt 2
		\item Schritt 3
		\item Schritt 4
	\end{itemize}
	\begin{overlayarea}{\textwidth}{1em}
		\only<2>{Dieser Hinweis wird nur für Schritt 2 angezeigt.}
		\only<3>{Dieser Hinweis wird nur für Schritt 3 angezeigt.}
	\end{overlayarea}
	Mit overlayarea verschiebt sich nachfolgender Text nicht.
\end{frame}

\begin{frame}{Beliebige Inhalte ein- und ausblenden}
	\begin{itemize}[<+->]
		\item Schritt 1
		\item Schritt 2
		\item Schritt 3
		\item Schritt 4
	\end{itemize}
	Inhalte können auch durch Animationen aufgedeckt werden.\\
	\uncover<2>{Dieser Hinweis wird nur für Schritt 2 angezeigt.}\\
	\uncover<3>{Dieser Hinweis wird nur für Schritt 3 angezeigt.}
\end{frame}

\section{Zeichnungen mit Animationen}
\begin{frame}{Animationen in Verbindung mit Zeichnungen}{Integration von Beamer mit tikz}
\begin{columns}[T]
	\begin{column}[T]{0.5\textwidth}
		\begin{center}
		\begin{tikzpicture}
			\useasboundingbox (-2,-2) rectangle (2,2);
			\node[draw,circle,fill=Senfgelb] (A) at (-1,0) {$A$}; 
			\node<3->[draw,circle,fill=Petrol,text=Senfgelb] (B) at ( 1,0) {$B$};
			\draw<5->[->] (A) to[out=45,in=135] (B);
			\draw<7->[->] (B) to[out=225,in=315] (A);			
		\end{tikzpicture}
		\end{center}
	\end{column}
	\begin{column}[T]{0.5\textwidth}
		\begin{itemize}
			\item Sei ein Knoten $A$ gegeben
			\item<2-> Wir fügen einen weiteren Knoten $B$ hinzu 
			\item<4-> Dann ziehen wir eine Kante von $A$ nach $B$
			\item<6-> ... und eine von $B$ nach $A$
		\end{itemize}
	\end{column}
\end{columns}	
\end{frame}

\section*{Schluss}
\begin{frame}
	\begin{center}
		\huge{Vielen Dank für die Aufmerksamkeit}
	\end{center}
	\begin{center}
		\Huge{Fragen?}
	\end{center}
\end{frame}

\begin{frame}[allowframebreaks]{\bibname}
\bibliography{tutorial}     %BibTeX-Datei literatur.bib
\end{frame}


\end{document}
