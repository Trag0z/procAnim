%Praesentationsmodus
\documentclass[t,aspectratio=169,divpsnames]{beamer}
%Die Beameroption aspectratio legt das verwendete Seitenverhaeltnis fest
%aspectratio=169	16:9 Seitenverhaeltnis
%aspectratio=1610	16:10 Seitenverhaeltnis
%aspectratio=43		4:3 Seitenverhaeltnis
%Die Beameroption envcountsect nummeriert Umgebungen wie theorem pro section durch.
%Die Beameroption divpsnames wird an das xcolor Paket durchgereicht.

%Handout-Generierung mit Foliennotizen (statt obiger Zeile für den Präsentationsmodus verwenden)
%\documentclass[t,handout,aspectratio=169]{beamer}
%\setbeameroption{show notes}

\usepackage[utf8]{inputenc}

% Deutsch
\usepackage[ngerman]{babel} 
\usepackage{bibgerm}

% Englisch
%\usepackage[english]{babel}

\input{templatesetup}
\usepackage{listings}
\lstset
{
	basicstyle=\ttfamily, 
	keywordstyle=\color{blue}\bfseries\ttfamily,
	identifierstyle=\ttfamily, 
	stringstyle=\ttfamily,
	commentstyle=\color{ForestGreen},
	showstringspaces=false,
	framexleftmargin=7mm, 
	breaklines=true,
	tabsize=3,
	showtabs=false,
	frame=single, 
	rulesepcolor=\color{blue},
	numbers=left,
	linewidth=146mm,
	xleftmargin=8mm,
	language={C++},
}

% Stil des Literaturverzeichnisses
\bibliographystyle{geralpha}
%\bibliographystyle{alpha}
%\bibliographystyle{abstract}

%Bitte ausfuellen:
\title{Entwicklung eines 2D Charakter-Animationssystemes für automatische Laufbewegungen}
\subtitle{Kolloquiumsvortrag}
\author{Daniel Track}
\institute{Hochschule Trier}
\date{03.11.2020}
\subject{Thema für PDF-Metadaten (optional)}

%Inhaltsverzeichnisses bis auf subsubsection-Ebene:
%\setcounter{tocdepth}{3}

%Aktivieren, um am Anfang jeder Section ein Inhaltsverzeichnis zur Section anzuzeigen
%\AtBeginSection[]
%{
%\begin{frame}<beamer>
%\frametitle{Agenda}
%\tableofcontents[currentsection,hideothersubsections,sectionstyle=show/hide,subsubsectionstyle=show/show]
%\end{frame}
%}

%Aktivieren, um alles Schritt-fuer-Schritt einzublenden
%\beamerdefaultoverlayspecification{<+->}

\begin{document}

\begin{frame}
    \titlepage
\end{frame}

\begin{frame}
    \frametitle{Agenda}
    \tableofcontents
    %\tableofcontents[hideallsubsections] % Subsections ausblenden
    %\tableofcontents[pausesections] %Sections Schritt-fuer-Schritt einblenden
\end{frame}

\section{Einleitung}

\begin{frame}{Motivation}
    \begin{itemize}
        \item Gängige 2D-Animationssysteme benötigen oft sehr viele Sprites % Gibt natürlich Ausnahmen/andere Systeme 
        \item Nicht einfach skalierbar für hohe Framerates ($>$ 60 fps)
        \item Skelettbasierte 3D-Animation hat diese Probleme nicht
        \item Außerdem flexibler, da Runtime-Daten in die Animation mit einbezogen werden können
    \end{itemize}

    $\Rightarrow$ Warum nicht das 3D-System auf 2D-Charaktere übertragen?
\end{frame}

\begin{frame}{Zielsetzung}{Anforderungen an das Aniamtionssystem}
    \begin{itemize}
        \item Generierung von glaubhaften Laufanimationen aus einem einzelnen PNG, gepaart mit einem Skelett
        \item Bewegung über verschieden hohe Untergründe durch Einbeziehung von Laufzeitdaten
        \item Responsive Steuerung
              \begin{itemize}
                  \item Schnelle Reaktion auf Inputs
                  \item Laufgeschwindigkeit mit Control-Stick präzise regulierbar
              \end{itemize}
        \item Anpassbarkeit des Animationsverhaltens
    \end{itemize}
\end{frame}


\section{Forschungsstand}

\section{Implementierung}

\begin{frame}{Programmaufbau}
    % hier ausschnitt aus UML?
\end{frame}

\begin{frame}{Animationsprozess}

\end{frame}

\section{Live Demonstration}
\begin{frame}{Live Demonstration}
    \begin{center}
        \huge{(Demonstration des Programms)}
    \end{center}
\end{frame}

\section*{Schluss}
\begin{frame}
    \begin{center}
        \huge{Vielen Dank für die Aufmerksamkeit}
    \end{center}
    \begin{center}
        \Huge{Fragen?}
    \end{center}
\end{frame}

\begin{frame}[allowframebreaks]{\bibname}
    \bibliography{literatur}     %BibTeX-Datei literatur.bib
\end{frame}


\end{document}
