\chapter{Grundlagen}

\section{Skelettbasierte Animation}
Die skelettbasierte Animation ist ein weit verbreitetes Verfahren zur Animation von 3D-Charakteren. Dabei werden für jeden Charakter zwei verschiedene Repräsentationen erstellt: Die erste besteht aus einem Dreiecksnetz (Skin), stellt die Oberfläche des Modells dar und wird zur visuellen Darstellung verwendet. Hinzu kommt eine Menge von Knochen, die nicht sichtbar sind und nur zur Animation des Dreiecksnetzes verwendet werden. Diese Knochen bilden eine hierarchische Datenstruktur und mit Ausnahme des Root-Knochens bestehen sie alle aus den folgenden drei Werten:

\begin{enumerate}
    \item Ein Parent-Knochen
    \item Eine dreidimensionale Transformationsmatrix relativ zum Parent (Bind-Pose Transform)
    \item Eine Rotation
\end{enumerate}

Der Root-Knochen hat keinen Parent, seine Transformationsmatrix stellt eine Transformation vom globalen Koordinatensystem ins Koordinatensystem des Knochens dar. Durch die Rotation der Knochen können neue Posen für den Charakter gebildet werden. Dabei wird die Rotation im lokalen Koordinatensystem des jeweiligen Knochens angewendet und beeinflusst so Position aller Knochen weiter unten in der Hierarchie. Jeder Vertex des Skins ist einem (Rigid Binding) oder mehreren Knochen (Smooth Binding) zugeordnet. Zur Bestimmung seiner neuen Position wird der Vertex in das Koordinatensystem der entsprechenden Knochen transformiert, dort rotiert und dann wieder ins Ursprungskoordinatensystem zurück transformiert. Um einen Vertex $v$ in eine beliebige Pose zu transformieren gelten die Formeln:

\begin{subequations}
    \begin{equation}
        \label{bone1}
        \hat{v} = \prod_{i=0}^{k} C_{i} (\prod_{i=0}^{k} B_{i})^{-1}p = J_{k}v
    \end{equation}
    \begin{equation}
        \label{bone2}
        \hat{v} = \sum_{i=0}^{n} w_{i} J_{i}v   \text{\hspace{.7cm} mit  } \sum_{i=0}^{n} w_{i} = 1
    \end{equation}
\end{subequations}

Dabei stellen $B_{i}^{-1}$ die Inverse Bind-Pose Transformationen der Knochen und $C_{i}$ die Transformationen für die aktuelle Pose dar. Die Gleichung in \ref{bone1} beschreibt das Verfahren beim Rigid Binding. Beim Smooth Binding (\ref{bone2}) werden Joint-Matrizen für alle relevanten Knochen berechnet, die dann mit deinem Gewichtungsfaktor $w_{i}$ auf den Vertex wirken.

\section{Inverse Kinematics}
Bei Inverse Kinematics handelt es sich um einen mathematischen Prozess zur Berechnung von Gelenkparametern, um das Ende einer kinematischen Kette (hier meist Arme oder Beine) an einem bestimmten Punkt zu positionieren. 

\section{Hermitekurven}