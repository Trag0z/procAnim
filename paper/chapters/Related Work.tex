\chapter{Related Work}

Obwohl die Idee des hier vorgestellten Animationssystems nicht gerade revolutionär wirkt und ihre Vorteile zumindest auf den ersten Blick auf der Hand zu liegen scheinen, konnten kaum wissenschaftlichen Arbeiten gefunden werden, die sich speziell mit dem Thema der skelettbasierten Animation von 2D-Charakteren in Videospielen beschäftigen.

\section{Skelettale Animation}
 [Für past oder present tense entscheiden!]
Einen der frühen Versuche, skelettbasierte Animation für die Bewegungen zweidimensionaler Charaktere zu verwenden, stellten Burtnyk und Wein \cite{burtnyk1976interactive} vor. Bis dahin wurden häufig Keyframes von Animationen gezeichnet und die restlichen Bilder durch Interpolation dieser Keyframes erzeugt. Dieses System erlaubte jedoch nur geradlinige Bewegung in gleich großen Schritten zwischen den Keyframes. Die Autoren führten ein Skelett in das Animationssystem ein, das es erlaubte, mit verhältnismäßig wenig Aufwand eine hohe Anzahl von Keyframes zu erstellen, da nur die simple Strichmännchen-Repräsentation dieser Knochen neu gezeichnet werden musste. Das finale Bild wird dann anhand dieser Knochen in entsprechende Posen verzerrt.

Van Overveld \cite{van1990technique} beschreibt ein Verfahren zur 2D-Animation vor, bei dem das Skelett vom Rest des Objekts getrennt wird. Das Skelett bzw. einige seiner Punkte werden dann prozedural in Echtzeit animiert und die Bewegungen der restlichen Punkte dynamisch simuliert. Die beschränkenden Kräfte auf das Skelett werden bei dieser Simulation erst nur approximiert und später dann korrigiert, um eine Performance zu erreichen, die schnell genug für eine Darstellung in Echtzeit ist.

Auf diesem Artikel aufbauend entwickelten Aoki et al. \cite{aoki1999dynamic} ein Verfahren zur realistischen Animation von Pflanzen, die vom Wind bewegt werden. Sie verwendeten dazu digital erstellte Bilder von Pflanzen und versahen sie per Hand mit Skeletten. Diesen Skeletten wurden dann mechanische Eigenschaften zugewiesen, mit denen die Simulation im letzten Schritt arbeitete. Sie beschreiben ihre Ergebnisse als sehr realistisch, mehr als zwei im Artikel enthaltene Beispielbilder konnten aber nicht gefunden werden.

Bruderlin et al. \cite{bruderlin1994procedural} entwickelten das \textit{Life Forms} System zur Animation von menschlichen Figuren im dreidimensionalen Raum. Sie verwenden dabei Inverse Kinematics, um die Figuren mit ihren Händen nach Objekten greifen zu lassen. Die Bewegungen der anderen daran beteiligten Körperteile wie Arm und Schulter werden dabei durch die Knochen und die für diese zuvor festgelegte Beschränkungen errechnet. Außerdem bietet ihre Anwendung die Möglichkeit zur prozeduralen Erstellung von Laufbewegungen. Dabei werden als Hauptparameter die Schrittweite und Schrittfrequenz (und daraus resultieren die Laufgeschwindigkeit) vom Nutzer angegeben. Das Programm erstellt dann vier Kontrollpunkte für einen Schritt in der Laufbewegung: Den Anfang, das Ende und dazwischen die beiden Zeitpunkte, an denen die Hüfte am höchsten bzw. am niedrigsten ist. Daraufhin wird eine kubische Spline-Kurve berechnet, die durch diese vier Punkte verläuft. Die Position der Hüfte des Modells wird dann entlang dieser Kurve interpoliert und die Stellung der Beine dazu passend berechnet.

Ein noch nicht lange zurückliegendes Paper von Pangesti und Kollegen \cite{pangesti2019analysis} beschreibt die Entwicklung eines Verfahrens zur einfacheren Erstellung von Laufanimationen für 2D-Charaktere mithilfe eines Skeletts und Inverse Kinematics. Außerdem war ein weiteres erklärtes Ziel ihrer Arbeit, eine Animation für das Umkehren während des Laufens zu erzeugen. Durch die Analyse von Laufbewegungen echter Menschen bestimmten sie die Freiheitsgrade der verschiedenen Knochen und wendeten diese auf das virtuelle Skelett an. So wird automatisiert verhindert, dass beispielsweise ein Ellenbogen in die falsche Richtung geknickt wird. Ein Animator kann dann Keyframes festlegen, denen der Körper des Modells folgen soll, ohne dass dadurch Fehlstellungen entstehen können. Die Arbeit demonstriert dieses Verfahren dann auch noch mit einer Animation für das Umkehren des Charakters während der Laufbewegung. Bemerkenswert dabei ist, dass das verwendete Modell dreidimensional ist und dann gerendert wird, um die Frames für die finale, zweidimensionale Animation zu erzeugen.

\section{Prozedurale Laufanimation}
Da Laufbewegungen einen der wohl häufigsten Bewegungsabläufe von Menschen darstellen, gibt es eine lange Reihen von Arbeiten zum Problem der Animation von menschlichen Laufbewegungen. In vielen Fällen werden die beschriebenen Animationen dabei einmalig erstellt und dann genau so abgespielt und aufgezeichnet (beispielsweise für computergenerierte Charaktere in einem Film). Da Videospiele ein interaktives Medium sind, ist es zur Anwendung in Spielen jedoch oft wichtig, dass die Animationen noch zur Laufzeit des Programms an das gewünschte Verhalten angepasst werden können. Grob können drei verschiedene Ansätze verwendet werden, um Bewegungen zur Laufzeit zu erstellen: Prozedurale Bewegung (auch algorithmische Bewegung), beispielbasierte Bewegung und dynamisch-simulierte Bewegung \cite{johansen2009automated}.

Systeme zur prozeduralen Bewegung verwenden ein Verständnis für die Bewegungsabläufe von Charakteren, um neue Bewegungen spontan generieren zu können. Beispielbasierte Ansätze hingegen schöpfen aus einer Reihe von zuvor erstellten Animationen und kombinieren diese, um neue Bewegungsabläufe zu erstellen. Bisweilen kommen dabei auch noch prozedurale Techniken zum Einsatz. Dynamische Simulationen hingegen berechnen ein physikalisches Modell des Charakters und seiner Umgebung und versuchen so, realistische Bewegungen zu erstellen. Als Grundlage können dabei auch Beispielanimationen oder prozedural erstellte Bewegungen verwendet werden.

Da dynamische Simulationen aufgrund ihrer physikalischen Modelle meist einen höheren Rechenaufwand mit sich bringen und außerdem nur realistische Bewegungsabläufe generieren können (was nicht Ziel dieser Arbeit ist – die Bewegungen sollen nur glaubhaft aussehen), werden diese Ansätze hier nicht weiter verfolgt. Ideen der beispielbasierten Animation, speziell die Interpolation zwischen mehreren zuvor erstellten Bewegungsabläufen, könnten dabei helfen, Laufbewegungen mehr individuellen Charakter zu geben und außerdem die Komplexität des prozeduralen Systems durch ihre Bewegungsvorgaben reduzieren.

Einen Ansatz zur prozeduralen Generierung von Laufbewegungen in Echtzeit beschrieben Bruderlin und Calvert \cite{bruderlin1993interactive} \cite{bruderlin1996knowledge}. Sie simulierten die grundlegende Bewegung der Beine als wären sie ein umgekehrtes Pendel und approximierten die Flugbahn des Torsos in der Luft mit einer Parabel. Ihr System bietet dem Nutzer eine große Anzahl von Variablen zur Beeinflussung der Animation und ermöglicht so viele verschiedene, ausdrucksstarke Bewegungen, trotz ihrer rein synthetischen Natur. Es sind aber nur Bewegungen auf ebenen Untergründen möglich.

