\chapter{Einleitung}
Menschen sind überall um uns herum. Die meisten von uns leben und arbeiten täglich mit anderen Mitgliedern der eigenen Spezies und so ist es kein Wunder, dass wir ihren Bewegungsabläufen extrem vertraut sind. Auch in Videospielen kontrolliert der Spieler zumeist Menschen oder zumindest humanoide Gestalten, die stark an sie angelehnt sind. Der enorme Menge an Erfahrungen mit menschlichen Bewegungsabläufen, die die meisten Spieler haben, sorgt dabei aber auch für große Erwartungen an die Repräsentation eben dieser Bewegungen in Spielen, da kleine Ungereimtheiten besonders schnell auffallen können. Das stellt das Feld der Computeranimation vor große Herausforderungen.

Diese Arbeit soll versuchen, durch die Entwicklung eines Animationssystems für zweidimensionale Charaktere einen Beitrag zum Überkommen dieser Herausforderungen zu leisten.

\section{Motivation}
Obwohl die enorme Leistungsfähigkeit moderner Computer es erlaubt, große, dreidimensionale Welten in Videospielen immer realistischer darstellen zu können, erscheinen noch immer sehr viele zweidimensionale Spiele, die das große Potenzial dieser Maschinen nicht ausreizen. Dafür mag es eine Reihe von Gründen geben, angefangen bei der Nostalgie vieler Spieler für die 2D-Spiele älterer Konsolengenerationen bis hin zu den meist wesentlich komplexeren mathematischen Verfahren, die zur Programmierung von dreidimensionalen Anwendungen verwendet werden. In vielen Fällen wird es jedoch auch damit zu tun haben, dass die Erstellung von 3D-Modellen und deren Animation besonders kleine Entwicklerstudios vor eine große Herausforderung stellt. 2D-Assets hingegen können recht leicht mit einem der vielen verfügbaren Bildbearbeitungsprogramme erstellt oder sogar per Hand gezeichnet und digitalisiert werden.

Einen entscheidenden Vorteil bieten 3D-Modelle jedoch gegenüber 2D-Sprites: Zur Animation werden 3D-Objekte meist auf ein Skelett gespannt, für dessen Knochen dann Zielstellungen angegeben werden. Das Modell folgt dann der Bewegung der Knochen. Bei 2D-Modellen hingegen wird häufig jeder einzelne Schritt der Animation zuvor erstellt. Auch hier gibt es Verfahren, mit denen Computer den Grafikdesignern Arbeit abnehmen können; trotzdem wird meist eine beachtliche Anzahl an Animationsframes per Hand erstellt. Viele der 2D-Spiele sind deshalb außerdem auf eine Framerate von 60 Bildern pro Sekunde limitiert. Monitore mit höheren Framerates werden aber zunehmend günstiger und populärer, und 120, 144 oder gar 240 FPS zu unterstützen stellt für die gängigen 2D-Animationssysteme eine schwierige Herausforderung dar, während es bei den meisten 3D-Anwendungen hauptsächlich eine Frage der Hardware ist.

Ein weiterer Vorteil der gängigen 3D-Animationsverfahren ist, dass genaue Zielpunkte für Bewegungen während der Laufzeit des Programms errechnet und Animationen dann so ausgeführt werden können, dass sie diese Punkte exakt treffen (z.B. wenn ein Charakter ein Objekte mit seiner Hand greift). Mit 2D-Animationen, deren Bilder vor der Laufzeit des Programms gezeichnet werden, ist dies nicht möglich.

Eine Übertragung des Verfahrens der skelettbasierten Animation auf 2D-Charaktere könnte dabei helfen, sowohl den Arbeitsaufwand bei der Erstellung von Animationen zu minimieren als auch neue Arten von Animationen zu ermöglichen, die erst zur Laufzeit des Programms gebildet werden.

\section{Zielsetzung}
Um die Vorteile der zuvor beschriebenen skelettbasierten Animierung von 2D-Charakteren zu demonstrieren, soll ein System erstellt werden, dass nur einen einzelnen Sprite eines Charakters verwendet und eine Laufanimation darstellt. Um weiterhin zu zeigen, welche Möglichkeiten das Einbeziehen von Laufzeitdaten eröffnet, soll der Charakter auch über Flächen verschiedener Höhen (wie z.B. eine Treppe hinauf) laufen können und dabei sowohl die Position, auf der der Fuß aufgesetzt wird als auch die Bewegung dorthin dynamisch zur Laufzeit errechnen.

Außerdem soll sichergestellt werden, dass die Steuerung des Charakters responsiv ist. Dazu gehört, dass er schnell auf Eingaben reagiert und seine Animation das auch angemessen dem Spieler kommuniziert. Ein weiterer Aspekt ist in diesem Fall aber auch, dass der Charakter sich nicht nur in einer vorbestimmten Geschwindigkeit bewegen kann. Die dynamische Animation eröffnet die Möglichkeit, die Länge und genaue Bewegungsabfolge jedes Schrittes neu zu bestimmen. Von dieser Möglichkeit soll gebrauch gemacht werden, um ein präzise Bestimmung der Laufgeschwindigkeit zu erlauben.

Da für verschiedene Charaktere auch verschiedene Animationsverhalten nötig sind, soll es Möglichkeiten zur Anpassung der Animationen geben. Dies könnte beispielsweise über die Anpassung der Parameter geschehen, die das System zur Generierung neuer Animationen nutzt.

Eine wichtige Limitierung dabei ist, dass ausdrücklich \textit{nicht} Ziel dieser Arbeit ist, realistische bzw. physikalisch korrekte Bewegungen zu zeigen. Auf den Charakter wirkt also keine Gravitation, er muss kein Gleichgewicht halten oder sein Momentum abbremsen. Die Bewegung soll jedoch glaubhaft\footnote{Im Sinne von Bates et al.\cite{bates1994role}: Glaubhaft ist ein Charakter, wenn das Publikum ihn akzeptiert, als wäre er lebendig. Realistisch ist er, wenn er der tatsächlichen Realität entspricht.} aussehen, kann dabei aber durchaus stilisiert sein, wie es in der Computeranimation üblich ist.

\section{Aufbau}
Im Folgenden werden zuerst einige der Grundlagen erläutert. Dabei handelt es sich um Techniken und Konzepte, die häufig in der Computeranimation zum Einsatz kommen und auch von dem hier vorgestellten System verwendet werden. Daraufhin legt Kapitel \ref{forschungsstand} den aktuellen Stand der Forschung in den relevanten Bereichen dar. Dabei wird sowohl auf die prozedurale generation von Laufanimationen als auch auf die Animation von 2D-Charakteren mithilfe von Skeletten eingegangen. Außerdem wird die hier vorliegende Arbeit in den Kontext der bestehenden Forschung eingeordnet.

Das entwickelte Animationssystem und dessen Implementierung wird dann in Kapitel \ref{implementierung} vorgestellt. Dazu werden zuerst konzeptionelle Überlegungen dargelegt, gefolgt von einer kurzen Erläuterung der vom Projekt verwendeten Programmbibliotheken. Abschnitt \ref{aufbau} beschreibt dann zunächst den generellen Aufbau des Programms, bevor in \ref{animator_section} die Details des entwickelten Animationssystems ausführlich erklärt werden.

Im darauf folgenden Kapitel werden die Ergebnisse besprochen und einige Beispiele von Animationen gezeigt, die mit dem System erstellt wurden. Abschließend wird in Kapitel \ref{zusammenfassung} dann ein Fazit gezogen und Möglichkeiten besprochen, wie die Arbeit noch verbessert werden kann.