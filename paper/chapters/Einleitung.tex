\chapter{Einleitung}
Diese Arbeit handelt von... (aber ist das genau so wie die Zusammenfassung am Anfang?)

[People are all around us. They inhabit our home, workplace, entertainment, and environment. Their presenceand actions are noted or ignored, enjoyed or disdained, analyzed or prescribed. The very ubiquitousness ofother people in our lives poses a tantalizing challenge to the computational modeler: people are at once themost common object of interest and yet the most structurally complex. from Simulating Humans: Computer Graphics,Animation, and Control]

\section{Motivation}
Obwohl die enorme Leistungsfähigkeit moderner Computer es erlaubt, große, dreidimensionale Welten in Videospielen immer realistischer darstellen zu können, erscheinen noch immer sehr viele zweidimensionale Spiele, die das große Potenzial dieser Maschinen nicht ausreizen. Dafür mag es eine Reihe von Gründen geben, angefangen bei der Nostalgie vieler Spieler für die 2D-Spiele älterer Konsolengenerationen bis hin zu den meist wesentlich komplexeren mathematischen Verfahren, die zur Programmierung von dreidimensionalen Anwendungen verwendet werden. In vielen Fällen wird es jedoch auch damit zu tun haben, dass die Erstellung von 3D-Modellen und deren Animation besonders kleine Entwicklerstudios vor eine große Herausforderung stellt. 2D-Assets hingegen können recht leicht mit einem der vielen verfügbaren Bildbearbeitungsprogramme erstellt oder sogar per Hand gezeichnet und digitalisiert werden.

Einen entscheidenden Vorteil bieten 3D-Modelle jedoch gegenüber 2D-Sprites: Zur Animation werden 3D-Objekte meist auf ein Skelett gespannt, für dessen Knochen dann Zielstellungen angegeben werden. Das Modell folgt dann der Bewegung der Knochen. Bei 2D-Modellen hingegen wird häufig jeder einzelne Schritt der Animation zuvor erstellt. Auch hier gibt es Verfahren, mit denen Computer den Grafikdesignern Arbeit abnehmen können; trotzdem wird meist eine beachtliche Anzahl an Animationsframes per Hand erstellt. Viele der 2D-Spiele sind deshalb außerdem auf eine Framerate von 60 Bildern pro Sekunde limitiert. Monitore mit höheren Framerates werden aber zunehmend günstiger und populärer, und 120, 144 oder gar 240 FPS zu unterstützen stellt für die gängigen 2D-Animationssysteme eine schwierige Herausforderung dar, während es bei den meisten 3D-Anwendungen hauptsächlich eine Frage der Hardware ist.

Ein weiterer Vorteil der gängigen 3D-Animationsverfahren ist, dass genaue Zielpunkte für Bewegungen während der Laufzeit des Programms errechnet und Animationen dann so ausgeführt werden können, dass sie diese Punkte exakt treffen (z.B. wenn ein Charakter ein Objekte mit seiner Hand greift). Mit 2D-Animationen, deren Bilder vor der Laufzeit des Programms gezeichnet werden, ist dies nicht möglich.

Eine Übertragung des Verfahrens der skelettbasierten Animation auf 2D-Charaktere könnte dabei helfen, sowohl den Arbeitsaufwand bei der Erstellung von Animationen zu minimieren als auch neue Arten von Animationen zu ermöglichen, die erst zur Laufzeit des Programms gebildet werden.

\section{Zielsetzung}
Um die Vorteile der zuvor beschriebenen skelettbasierten Animierung von 2D-Charakteren zu demonstrieren, soll ein System erstellt werden, dass nur einen einzelnen Sprite eines Charakters verwendet und eine Laufanimation darstellt. Um weiterhin zu zeigen, welche Möglichkeiten das Einbeziehen von Laufzeitdaten eröffnet, soll der Charakter auch über Flächen verschiedener Höhen (wie z.B. eine Treppe hinauf) laufen können und dabei sowohl die Position, auf der der Fuß aufgesetzt wird als auch die Bewegung dorthin dynamisch zur Laufzeit errechnen.

Eine wichtige Limitierung dabei ist, dass ausdrücklich \textit{nicht} Ziel dieser Arbeit ist, physikalisch korrekte Bewegungen zu zeigen. Auf den Charakter wirkt also keine Gravitation, er muss kein Gleichgewicht halten oder sein Momentum abbremsen. Die Bewegung soll jedoch glaubhaft aussehen, kann dabei aber durchaus stilisiert sein, wie es in der Computeranimation üblich ist.

\section{Limitierungen}

\section{Aufbau}