\kurzfassung

%% deutsch
\paragraph*{}
Diese Arbeit beschreibt die Entwicklung eines Systems, das die dynamische Animation der Laufbewegungen von zweidimensionalen Charakteren in Videospielen ermöglicht. Dabei wird zuerst ein Charaktermodell als Skelett erstellt und mit einer Textur versehen. Das Programm erstellt dann anhand der Eingaben des Nutzers und der Daten des Skeletts neue Bewegungsabläufe für jeden einzelnen Schritt des Charakters. Die Bewegungsgeschwindigkeit der so erstellten Animationen ist präzise auf die Eingabe des Spielers abgestimmt und erlaubt Bewegungen über Untergründe verschiedener Höhen.

Dazu werden Hermite Splines für die Hände, Füße und das Becken des Charakters berechnet, denen die entsprechenden Punkte des Körpers dann folgen. Die Splines sind an den Boden unter dem Spieler angepasst und variieren die Schrittweite und -form je nach geforderter Bewegungsgeschwindigkeit.
Außerdem werden die Splines basierend auf einer Reihe von Spline-Prototypen gebildet, die zuvor vom Nutzer festgelegt werden können. So soll eine große Anpassbarkeit der Animationen an verschiedene Charaktere und Anwendungsfälle ermöglicht werden.
Es werden einige Beispiele von Animationen gezeigt, die von dem System erstellt wurden, um die Variation der Bewegungen und den Umgang mit verschiedenen Situationen zu demonstrieren.


%% englisch
\paragraph*{}
This thesis describes the development of a system for the dynamic animation of walking movements for two-dimensional characters in video games. To achieve this, the character model is created as a skeleton and provided with a texture. The application then uses the user's input and the data of the skeleton to generate new animations for every single step of the character. Animations created this way allow for movements with a speed that is fitted precisely to the player's input and that are also able to adapt to differences in ground height.

In order to do this, the system creates Hermite splines for the hands, feet and pelvis of the character that the respective body parts follow. These splines are adjusted to the floor below the player and vary the step distance and shape depending on the required walking speed.
Furthermore, these splines are based on a set of prototype splines that the user may set beforehand. This allows for customization of the created animations in order to fit them to different characters and use cases.
The thesis shows some examples of animations created with the presented system to showcase different possible variations of movements and the adaptation to various situations.