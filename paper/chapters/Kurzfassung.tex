\kurzfassung

%% deutsch
\paragraph*{}
Diese Arbeit beschreibt die Entwicklung eines Systems, das die dynamische Animation der Laufbewegungen von zweidimensionalen Charakteren ermöglicht. Dabei wird zuerst ein Charaktermodell als Skelett erstellt und mit einer Textur versehen. Das Programm erstellt dann anhand der Eingaben des Nutzers und der Daten des Skeletts neue Bewegungsabläufe für jeden einzelnen Schritt des Charakters. Die so erstellten Animationen ermöglichen Bewegungen in präzise an die Eingabe des Spielers angepasster Geschwindigkeit und über Untergründe verschiedener Höhen. Dazu werden Hermite Splines für die Hände, Füße und das Becken des Charakters berechnet, denen die entsprechenden Punkte des Körpers dann folgen. Die Splines sind an den Boden unter dem Spieler angepasst sind und variieren die Schrittweite und -form je nach geforderter Bewegungsgeschwindigkeit.
Außerdem werden die Splines basierend auf einer Reihe von Spline-Prototypen gebildet, die zuvor vom Nutzer festgelegt werden können. So soll eine große Anpassbarkeit der Animationen an verschiedene Charaktere und Anwendungsfälle ermöglicht werden.
Es werden einige Beispiele von Animationen gezeigt, die von dem System erstellt wurden, um die Variation der Bewegungen und den Umgang mit verschiedenen Situationen zu demonstrieren.


%% englisch
\paragraph*{}
The same in english.
