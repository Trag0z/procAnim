\chapter{Zusammenfassung und Ausblick}

In diesem Kapitel soll die Arbeit noch einmal kurz zusammengefasst werden. Insbesondere sollen die wesentlichen Ergebnisse Ihrer Arbeit herausgehoben werden. Erfahrungen, die z.B. Benutzer mit der Mensch-Maschine-Schnittstelle gemacht haben oder Ergebnisse von Leistungsmessungen sollen an dieser Stelle pr�sentiert werden. Sie k�nnen in diesem Kapitel auch die Ergebnisse oder das Arbeitsumfeld Ihrer Arbeit kritisch bewerten. W�nschenswerte Erweiterungen sollen als Hinweise auf weiterf�hrende Arbeiten erw�hnt werden.



Limits:
- kein flight-state beim rennen
- keine physikalische simulation
- keine verlangsamung/beschleunigung bei bergauf/bergab gehen
- kein kopf bone, ist aber evtl unnötig

- shape deformation des charakters kann hart sein, kein fancy algorithmus dafür wird angewendet
> wäre aber auch cpu/gpu intensiv, also für spiele lieber nicht? würde zumindest viel optimierungsarbeit kosten
- keine schrägen oberflächen, collision detection nur aabb
- keine foot-base
- zyklische bewegungen
- nur humanoide, könnte aber erweitert werden

Verbesserungen:
- punkte festlegen, durch die die splines verlaufen sollen (siehe bruderlin1994procedural)
- mehrere schritte in die zukunft simulieren, um besser auf hindernisse reagieren zu können (siehe Animation of Human Walking in Virtual Environments)