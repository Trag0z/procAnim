\chapter{Implementierung}
In diesem Kapitel wird der Vorgang der Implementierung erläutert. Dabei wird zuerst...


\section{Konzeption}
Da der hauptsächliche Anwendungsbereich des in dieser Arbeit beschriebenen Animationssystems Videospiele sind, werden zur Implementierung auch Werkzeuge verwendet, die in der Spieleindustrie gängig sind. Programmiert wird deshalb in C++ unter Windows und mit einem x64-Prozessor als Zielplatform. Um gute Performance des Programms zu gewährleisten soll ein möglichst großer Teil der Berechnungen auf der Grafikkarte ausgeführt werden. Dafür kommt OpenGL als Grafikbibliothek zum Einsatz und es werden Shader in GLSL geschrieben.

mit gamepad als bevorzugte steuerungsmethode implementiert


\section{Verwendete Bibliotheken}
Um viele der grundlegenden Prozesse im Programm zu vereinfachen und schon gut gelöste Probleme wie z.B. das Erstellen eines OpenGL-Kontexts oder das Laden der Modelldaten nicht erneut lösen zu müssen, werden einige Bibliotheken verwendet. Diese werden im Folgenden im Detail erläutert.

\subsection{OpenGL}
\href{https://www.opengl.org/}{OpenGL} ist eins der meist verwendeten Grafik-APIs und bietet als solches diverse Funktionen, um mit der Grafikkarte zu interagieren. Es bietet außerdem die Shading-Language GLSL, die zur Programmierung der Shader verwendet wird. In dieser Anwendung kommt die Version 3.3 Core zum Einsatz.

\subsection{OpenGL Mathematics (GLM)}
Bei \href{https://glm.g-truc.net/0.9.9/index.html}{GLM} handelt es sich um eine Mathematik-Bibliothek für C++, die der Namensgebung und Funktionalität von GLSL entspricht und deren Datenformate so aufgebaut sind, dass sie den von GLSL erwarteten Formaten entsprechen. Die von GLM bereitgestellten Matrix- und Vektorklassen und die Funktionen, die mit ihnen arbeiten, werden für Berechnungen auf der CPU sowie zum Hochladen von Daten in den GPU-Speicher verwendet.

\subsection{Simple DirectMedia Layer (SDL)}
\href{https://www.libsdl.org/index.php}{SDL} ist eine leichtgewichtige C-Bibliothek, die das Erstellen und Verwalten eines Fensters und dazugehörigen OpenGL-Kontexts unter Windows stark vereinfacht. Außerdem bietet sie mit simplen Funktionen Zugang zu Maus-, Tastatur- und Gamepad-Inputs. Des Weiteren wird die dazugehörige Bibliothek \href{https://www.libsdl.org/projects/SDL_image/}{SDL_image} zum Laden von Bildern im PNG-Format verwendet.

\subsection{Open Asset Import Library (Assimp)}
Das Skelett des zu animierenden Charakters und damit zusammenhängend auch das Mesh sowie die UV-Koordinaten und Bone-Weights liegen dem Programm im FBX-Format vor.\href{http://assimp.org/}{Assimp} ermöglicht es, dieses Dateiformat zu parsen und dabei einige Post-Processing Operationen wie das Vereinigen identischer Vertices durchzuführen.

\subsection{Dear ImGui}
Bei \href{https://github.com/ocornut/imgui}{Dear ImGui} handelt es sich um eine Implementierung des Immediate Mode GUI Paradigmas. Es ermöglicht die Darstellung von grafischen Interfaces im bestehenden OpenGL-Kontext und wird hier verwendet, um ein Debug-Informationen anzuzeigen, die erstellten Tools zu kontrollieren und einige Variablen der Simulation zur Laufzeit ändern zu können.

\section{Aufbau des Programms}
Das UML-Klassendiagramm in \ref{} zeigt den generellen Aufbau der Anwendung. Im Zentrum steht dabei die Klasse \lstinline{Game}, die auf hoher Ebene den Ablauf koordiniert und die verschiedenen Komponenten der Simulation verwaltet. Ihre Methode \lstinline{init()} initialisiert alle nötigen Systeme und erstellt das Fenster, sodass das Programm in einem geeigneten Ausgangszustand ist. Die Methode \lstinline{run()} berechnet den nächsten Simulationsschritt und zeichnet das Ergebnis auf den Bildschirm.

Zum Zeichnen der Grafiken und Verwalten der Shader wird \lstinline{Renderer} verwendet. Er lädt bei seiner Initialisierung die drei verschiedenen Shader-Programme, die in der Anwendung zum Einsatz kommen und besitzt jeweils ein Objekt der drei entsprechenden Shader-Klassen \lstinline{RiggedShader}, \lstinline{TexturedShader} und \lstinline{DebugShader}. Die verschiedenen Shader und ihre Anwendungsgebiete werden in Abschnitt \ref{rendering_section} ausführlicher beschrieben.

Die Klassen \lstinline {MouseKeyboardInput} und \lstinline{Gamepad} verwalten die verschiedenen Eingabemöglichkeiten und werden zu Beginn jedes Frames aktualisiert. Bei \lstinline{Background} handelt es sich um eine sehr einfache Klasse, die nur dafür zuständig ist, eine Textur als Hintergrund zu rendern. Der Hintergrund stellt einen Referenzpunkt für die Bewegungen des Charakters da und erleichtert so die Beurteilung der Animationen.

Da der Charakter über Untergründe verschiedener Höhen laufen können soll, wurde die Klasse \lstinline{Level} zur Repräsentation der Bodenstruktur erstellt. Der Untergrund besteht dabei aus einer Reihe von Collidern, die jeweils aus Quadraten bestehen und entlang der globalen X- und Y-Achsen ausgerichtet sind. Um den Aufbau des Levels anzupassen gibt es außerdem den \lstinline{LevelEditor}. Er bietet ein simples User Interface zum Erstellen, Löschen und Verändern der Levelelemente. Weiterhin besteht die Möglichkeit, die Level in einem selbst erstellten Dateiformat zu speichern und wieder zu laden. Beim Start des Programms wird immer die Datei \lstinline{assets/default.level} geladen, über den Editor können aber auch noch weitere Levels zum testen verschiedener Aspekte der Laufsimulation geladen werden.

\lstinline{Player} repräsentiert den Spielercharakter und ist eine Unterklasse von \lstinline{Entity}. Er besitzt ein \lstinline{Mesh}, das wiederum aus einer Reihe von Vertices und Knochen besteht. Zusammen mit der Textur bietet es die Grundlage für die visuelle Darstellung des Charakters. Kontrolliert wird letzterer hauptsächlich mit den Inputdaten des \lstinline{Gamepad}.

Der \lstinline{Animator} steuert die Bewegungen des Spielercharakters und bildet somit den Kern der Anwendung. Er beinhaltet Repräsentationen der vier Gliedmaßen des Spielers in \lstinline{limbs} und Splines, an denen die Bewegungsabläufe orientiert werden in \lstinline{spline_prototypes}. Der genaue Ablauf des Animationsprozesses wird im Abschnitt \ref{animator_section} ausführlich erklärt.


\subsection{Rendering und Shader} \label{rendering_section}
Alle Klassen, die grafische Objekte in der Simulation darstellen, haben eine Methode \lstinline{void render(const Renderer&)} (eventuell noch mit weiteren Parametern), mit der sie sich selbst auf den Bildschirm zeichnen. Die Objekte wählen dann einen der drei verschiedenen Shader aus, indem sie auf dem entsprechenden Member-Objekt des Renderers \lstinline{use()} aufrufen und setzen gegebenenfalls noch Uniform-Variablen. Die zu rendernden Vertexdaten halten die Objekte in einer Instanziierung des Templates \lstinline{VertexArray<>} mit dem zum Shader passenden Vertextyp als Template-Argument. Sie rufen auf diesem VertexArray \lstinline{draw(GLenum mode)} auf, um die Vertices im entsprechenden Modus zeichnen zu lassen.

Die drei verfügbaren Shader sind:
\begin{enumerate}
    \item \lstinline{DebugShader}
    \item \lstinline{TexturedShader}
    \item \lstinline{RiggedShader}
\end{enumerate}

Der Debug-Shader wird hauptsächlich zur Visualisierung von Debug-Informationen verwendet. Es kann eine Farbe für die zu rendernden Primitive in Form einer Uniform-Variable eingestellt werden; Texturen werden nicht verwendet. Das Vertexformat dieses Shaders besteht nur aus einem zweidimensionalen Vektor zur Angabe der Vertexposition. Außerdem gibt es die statische Variable \lstinline{DEFAULT_VAO}, die ein simples Vertex Array zur Darstellung von quadratischen Objekten anbietet. Diese müssen so nicht ihre eigenen Vertices im Arbeitsspeicher der GPU verwalten, sondern können die beschriebenen Vertices verwenden und mithilfe der Model-Matrix beliebig transformieren.

Der Textured-Shader kommt beim Rendering von texturierten Objekten zum Einsatz. Seine Vertices enthalten deshalb neben der Position noch eine Texturkoordinate.

Nur Animator, Mesh, Spline und Background

\section{Animator} \label{animator_section}

\subsection{Anpassung der Bewegungen}