%%%%%%%%%%%%%%%%%%% vorlage.tex %%%%%%%%%%%%%%%%%%%%%%%%%%%%%
%
% LaTeX-Vorlage zur Erstellung von Projekt-Dokumentationen
% im Fachbereich Informatik der Hochschule Trier
%
% Basis: Vorlage svmono des Springer Verlags
%
%%%%%%%%%%%%%%%%%%%%%%%%%%%%%%%%%%%%%%%%%%%%%%%%%%%%%%%%%%%%%

\documentclass[envcountsame,envcountchap, deutsch]{i-studis}

\usepackage{makeidx}         	% Index
\usepackage{multicol}        	% Zweispaltiger Index
%\usepackage[bottom]{footmisc}	% Erzeugung von Fu�noten

%%-----------------------------------------------------
%\newif\ifpdf
%\ifx\pdfoutput\undefined
%\pdffalse
%\else
%\pdfoutput=1
%\pdftrue
%\fi
%%--------------------------------------------------------
%\ifpdf
\usepackage[pdftex]{graphicx}
\usepackage{epstopdf}
\usepackage[pdftex,plainpages=false]{hyperref}
%\else
%\usepackage{graphicx}
%\usepackage[plainpages=false]{hyperref}
%\fi

%%-----------------------------------------------------
\usepackage{color}				% Farbverwaltung
%\usepackage{ngerman} 			% Neue deutsche Rechtsschreibung
\usepackage[english, ngerman]{babel}
%\usepackage[latin1]{inputenc} 	% Erm�glicht Umlaute-Darstellung
\usepackage[utf8]{inputenc}  	% Erm�glicht Umlaute-Darstellung unter Linux (je nach verwendetem Format)

%-----------------------------------------------------
\usepackage{listings} 			% Code-Darstellung
\lstset
{
	basicstyle=\scriptsize, 	% print whole listing small
	keywordstyle=\color{blue}\bfseries,
								% underlined bold black keywords
	identifierstyle=, 			% nothing happens
	commentstyle=\color{red}, 	% white comments
	stringstyle=\ttfamily, 		% typewriter type for strings
	showstringspaces=false, 	% no special string spaces
	framexleftmargin=7mm, 
	tabsize=3,
	showtabs=false,
	frame=single, 
	rulesepcolor=\color{blue},
	numbers=left,
	linewidth=146mm,
	xleftmargin=8mm
}
\usepackage{textcomp} 			% Celsius-Darstellung
\usepackage{amssymb,amsfonts,amstext,amsmath}	% Mathematische Symbole
\usepackage[german, ruled, vlined]{algorithm2e}
\usepackage[a4paper]{geometry} % Andere Formatierung
\usepackage{bibgerm}
\usepackage{array}
\hyphenation{Ele-men-tar-ob-jek-te  ab-ge-tas-tet Aus-wer-tung House-holder-Matrix Le-ast-Squa-res-Al-go-ri-th-men} 		% Weitere Silbentrennung bei Bedarf angeben
\setlength{\textheight}{1.1\textheight}
\pagestyle{myheadings} 			% Erzeugt selbstdefinierte Kopfzeile
\makeindex 						% Index-Erstellung


%--------------------------------------------------------------------------
\begin{document}
%------------------------- Titelblatt -------------------------------------
\title{Entwicklung eines 2D Character-Animationssystemes für automatische Laufbewegungen}
\subtitle{Development of a 2D Character Animation System for automatic Walking}
%---- Die Art der Dokumentation kann hier ausgew�hlt werden---------------
%\project{Bachelor-Projektarbeit}
\project{Bachelor-Abschlussarbeit}
%\project{Master-Projektstudium}
%\project{Master-Abschlussarbeit}
%\project{Seminar zur Vorlesung ...}
%\project{Hausarbeit zur Vorlesung ...}
%--------------------------------------------------------------------------
\supervisor{Titel Vorname Name} 		% Betreuer der Arbeit
\author{Daniel Track} 							% Autor der Arbeit
\address{Ort,} 							% Im Zusammenhang mit dem Datum wird hinter dem Ort ein Komma angegeben
\submitdate{Abgabedatum} 				% Abgabedatum
%\begingroup
%  \renewcommand{\thepage}{title}
%  \mytitlepage
%  \newpage
%\endgroup
\begingroup
\renewcommand{\thepage}{Titel}
\mytitlepage
\newpage
\endgroup
%--------------------------------------------------------------------------
\frontmatter
%--------------------------------------------------------------------------
\preface

Ein Vorwort ist nicht unbedingt n�tig. Falls Sie ein Vorwort schreiben, so ist dies der Platz, um z.B. die Firma vorzustellen, in der diese Arbeit entstanden ist, oder einigen Leuten zu danken, die in irgendeiner Form positiv zur Entstehung dieser Arbeit beigetragen haben. Auf keinen Fall sollten Sie im Vorwort die Aufgabenstellung n�her erl�utern oder vertieft auf technische Sachverhalte eingehen.				% Vorwort (optional)
\kurzfassung

%% deutsch
\paragraph*{}
Diese Arbeit beschreibt die Entwicklung eines Systems, das die dynamische Animation der Laufbewegungen von zweidimensionalen Charakteren in Videospielen ermöglicht. Dabei wird zuerst ein Charaktermodell als Skelett erstellt und mit einer Textur versehen. Das Programm erstellt dann anhand der Eingaben des Nutzers und der Daten des Skeletts neue Bewegungsabläufe für jeden einzelnen Schritt des Charakters. Die Bewegungsgeschwindigkeit der so erstellten Animationen sind präzise auf die Eingabe des Spielers abgestimmt und erlaubt Bewegungen über Untergründe verschiedener Höhen.

Dazu werden Hermite Splines für die Hände, Füße und das Becken des Charakters berechnet, denen die entsprechenden Punkte des Körpers dann folgen. Die Splines sind an den Boden unter dem Spieler angepasst und variieren die Schrittweite und -form je nach geforderter Bewegungsgeschwindigkeit.
Außerdem werden die Splines basierend auf einer Reihe von Spline-Prototypen gebildet, die zuvor vom Nutzer festgelegt werden können. So soll eine große Anpassbarkeit der Animationen an verschiedene Charaktere und Anwendungsfälle ermöglicht werden.
Es werden einige Beispiele von Animationen gezeigt, die von dem System erstellt wurden, um die Variation der Bewegungen und den Umgang mit verschiedenen Situationen zu demonstrieren.


%% englisch
\paragraph*{}
This thesis describes the development of a system for the dynamic animation of walking movements for two-dimensional characters in video games. To achieve this, the character model is created as a skeleton and provided with a texture. The application then uses the user's input and the data of the skeleton to generate new animations for every single step of the character. Animations created this way allow for movements with a speed that is fitted precisely to the player's input and that are also able to adapt to differences in ground height.

In order to do this, the system creates Hermite splines for the hands, feet and pelvis of the character that the respective body parts follow. These splines are adjusted to the floor below the player and vary the step distance and shape depending on the required walking speed.
Furthermore, these splines are based on a set of prototype splines that the user may set beforehand. This allows for customization of the created animations in order to fit them to different characters and use cases.
The thesis shows some examples of animations created with the presented system to showcase different possible variations of movements and the adaptation to various situations. 			% Kurzfassung Deutsch/English
\tableofcontents 						% Inhaltsverzeichnis
\listoffigures 							% Abbildungsverzeichnis (optional)
\listoftables 							% Tabellenverzeichnis (optional)
%--------------------------------------------------------------------------
\mainmatter                        		% Hauptteil (ab hier arab. Seitenzahlen)
%--------------------------------------------------------------------------
% Die Kapitel werden in separaten .tex-Dateien abgelegt und hier eingebunden.
\chapter{Einleitung}
Diese Arbeit handelt von... (aber ist das genau so wie die Zusammenfassung am Anfang?)

[People are all around us. They inhabit our home, workplace, entertainment, and environment. Their presenceand actions are noted or ignored, enjoyed or disdained, analyzed or prescribed. The very ubiquitousness ofother people in our lives poses a tantalizing challenge to the computational modeler: people are at once themost common object of interest and yet the most structurally complex. from Simulating Humans: Computer Graphics,Animation, and Control]

\section{Motivation}
Obwohl die enorme Leistungsfähigkeit moderner Computer es erlaubt, große, dreidimensionale Welten in Videospielen immer realistischer darstellen zu können, erscheinen noch immer sehr viele zweidimensionale Spiele, die das große Potenzial dieser Maschinen nicht ausreizen. Dafür mag es eine Reihe von Gründen geben, angefangen bei der Nostalgie vieler Spieler für die 2D-Spiele älterer Konsolengenerationen bis hin zu den meist wesentlich komplexeren mathematischen Verfahren, die zur Programmierung von dreidimensionalen Anwendungen verwendet werden. In vielen Fällen wird es jedoch auch damit zu tun haben, dass die Erstellung von 3D-Modellen und deren Animation besonders kleine Entwicklerstudios vor eine große Herausforderung stellt. 2D-Assets hingegen können recht leicht mit einem der vielen verfügbaren Bildbearbeitungsprogramme erstellt oder sogar per Hand gezeichnet und digitalisiert werden.

Einen entscheidenden Vorteil bieten 3D-Modelle jedoch gegenüber 2D-Sprites: Zur Animation werden 3D-Objekte meist auf ein Skelett gespannt, für dessen Knochen dann Zielstellungen angegeben werden. Das Modell folgt dann der Bewegung der Knochen. Bei 2D-Modellen hingegen wird häufig jeder einzelne Schritt der Animation zuvor erstellt. Auch hier gibt es Verfahren, mit denen Computer den Grafikdesignern Arbeit abnehmen können; trotzdem wird meist eine beachtliche Anzahl an Animationsframes per Hand erstellt. Viele der 2D-Spiele sind deshalb außerdem auf eine Framerate von 60 Bildern pro Sekunde limitiert. Monitore mit höheren Framerates werden aber zunehmend günstiger und populärer, und 120, 144 oder gar 240 FPS zu unterstützen stellt für die gängigen 2D-Animationssysteme eine schwierige Herausforderung dar, während es bei den meisten 3D-Anwendungen hauptsächlich eine Frage der Hardware ist.

Ein weiterer Vorteil der gängigen 3D-Animationsverfahren ist, dass genaue Zielpunkte für Bewegungen während der Laufzeit des Programms errechnet und Animationen dann so ausgeführt werden können, dass sie diese Punkte exakt treffen (z.B. wenn ein Charakter ein Objekte mit seiner Hand greift). Mit 2D-Animationen, deren Bilder vor der Laufzeit des Programms gezeichnet werden, ist dies nicht möglich.

Eine Übertragung des Verfahrens der skelettbasierten Animation auf 2D-Charaktere könnte dabei helfen, sowohl den Arbeitsaufwand bei der Erstellung von Animationen zu minimieren als auch neue Arten von Animationen zu ermöglichen, die erst zur Laufzeit des Programms gebildet werden.

\section{Zielsetzung}
Um die Vorteile der zuvor beschriebenen skelettbasierten Animierung von 2D-Charakteren zu demonstrieren, soll ein System erstellt werden, dass nur einen einzelnen Sprite eines Charakters verwendet und eine Laufanimation darstellt. Um weiterhin zu zeigen, welche Möglichkeiten das Einbeziehen von Laufzeitdaten eröffnet, soll der Charakter auch über Flächen verschiedener Höhen (wie z.B. eine Treppe hinauf) laufen können und dabei sowohl die Position, auf der der Fuß aufgesetzt wird als auch die Bewegung dorthin dynamisch zur Laufzeit errechnen.

Eine wichtige Limitierung dabei ist, dass ausdrücklich \textit{nicht} Ziel dieser Arbeit ist, physikalisch korrekte Bewegungen zu zeigen. Auf den Charakter wirkt also keine Gravitation, er muss kein Gleichgewicht halten oder sein Momentum abbremsen. Die Bewegung soll jedoch glaubhaft aussehen, kann dabei aber durchaus stilisiert sein, wie es in der Computeranimation üblich ist.

\section{Aufbau}
\chapter{Grundlagen}

\section{Skelettbasierte Animation}

\section{Hermitekurven}
\chapter{Zusammenfassung und Ausblick}
In dieser Arbeit wurde versucht, ein System zu entwickeln, das die automatische Laufanimation von zweidimensionalen Charakteren zur Laufzeit des Programms anhand eines Skeletts und eines einzelnen Sprites ermöglicht. Dieses Ziel wurde erreicht; der Charakter kann sich dabei auch über verschieden hohe Untergründe und in beliebiger Geschwindigkeit bewegen. Außerdem reagiert der Charakter dabei meist schnell auf Eingaben und der Rechenaufwand der Simulation ist überschaubar, was die Verwendung in einem Videospiel begünstigt. Ein Editor für die zur Animation verwendeten Splines ermöglicht die Anpassung der Bewegungsabläufe an die Bedürfnisse des Nutzers.

Das System verfügt jedoch auch noch über viele Limitierungen, an denen zukünftige Arbeiten ansetzen könnten. Beispielsweise ist aktuell immer ein Fuß fest an den Boden gebunden, obwohl Menschen beim Rennen in der Regel zeitweise keinen Kontakt mehr zum Bode haben. Ein offensichtlichen Fehler ist außerdem die fehlende Collision Detection beim Bewegen über unebenen Untergrund. So kann es häufig vorkommen, dass sich beispielsweise der Fuß des Charakters durch ein Hindernis bewegt. Durch die Verwendung passender Splines lässt sich das Problem etwas minimieren, was aber die Anpassungsmöglichkeiten der Animation stark einschränkt.

Im Zusammenhang mit dieser Problematik der Kollisionsvermeidung wäre es möglicherweise hilfreich, eine komplexere Art von Splines zu verwenden. Die aktuellen Hermite Splines bestehen immer nur aus zwei Punkten und den zwei dazugehörigen Tangenten, ein dritter Punkt könnte jedoch beispielsweise an den Mittelpunkt des Schritts gesetzt werden, um die Bewegung besser zu definieren und um Hindernisse herum zu navigieren. Ein komplexeres System zur Anpassung der Tangenten in diesem dritten Punkt wäre dann aber vermutlich auch vonnöten, um bei Schritten auf höhere oder niedrigere Ebenen einen gutaussehenden Bewegungsablauf zu wahren.

Außerdem funktioniert der Algorithmus zum Finden der neuen Fußpositionen nur mit Collidern, die entlang der X- und Y-Achsen der Spielwelt ausgerichtet sind. Schräge Flächen sind also nicht möglich. Hier könnte das System leicht erweitert werden, um mehr Variation in den Levels zu erlauben.




% In diesem Kapitel soll die Arbeit noch einmal kurz zusammengefasst werden. Insbesondere sollen die wesentlichen Ergebnisse Ihrer Arbeit herausgehoben werden. Erfahrungen, die z.B. Benutzer mit der Mensch-Maschine-Schnittstelle gemacht haben oder Ergebnisse von Leistungsmessungen sollen an dieser Stelle pr�sentiert werden. Sie k�nnen in diesem Kapitel auch die Ergebnisse oder das Arbeitsumfeld Ihrer Arbeit kritisch bewerten. W�nschenswerte Erweiterungen sollen als Hinweise auf weiterf�hrende Arbeiten erw�hnt werden.



% Limits:
% - kein flight-state beim rennen
% - keine physikalische simulation
% - keine verlangsamung/beschleunigung bei bergauf/bergab gehen
% - kein kopf bone, ist aber evtl unnötig

% - shape deformation des charakters kann hart sein, kein fancy algorithmus dafür wird angewendet
% > wäre aber auch cpu/gpu intensiv, also für spiele lieber nicht? würde zumindest viel optimierungsarbeit kosten
% % - keine schrägen oberflächen, collision detection nur aabb
% - keine foot-base
% - zyklische bewegungen
% - nur humanoide, könnte aber erweitert werden
% % - collision derection nicht trivial, bei IK sogar extra hart wenn man alles auf einmal beachten will (also z.B. knie in den boden bei laufen)
% - umkehren sieht kacke aus, keine umkehranimation

% - bleibt der charakter stehen, sieht die aimation etwas komsich aus. dafür mehr responsive als wenn er zurück oder weiter steppen würde
% - je nachdem wann input gepollt wird kann erster schritt oft sehr klein sein

% Verbesserungen:
% - punkte festlegen, durch die die splines verlaufen sollen (siehe bruderlin1994procedural)
% - mehrere schritte in die zukunft simulieren, um besser auf hindernisse reagieren zu können (siehe Animation of Human Walking in Virtual Environments)
% ...
%--------------------------------------------------------------------------
\backmatter                        		% Anhang
%-------------------------------------------------------------------------
\bibliographystyle{geralpha}			% Literaturverzeichnis
\bibliography{literatur}     			% BibTeX-File literatur.bib
%--------------------------------------------------------------------------
\printindex 							% Index (optional)
%--------------------------------------------------------------------------
\begin{appendix}						% Anh�nge sind i.d.R. optional
    % \chapter{Glossar}

\abbreviation{DisASTer}		{DisASTer (Distributed Algorithms Simulation Terrain), A platform for the Implementation of Distributed Algorithms}
\abbreviation{DSM}			{Distributed Shared Memory}
\abbreviation{AC}			{Linearisierbarkeit (atomic consistency)}
\abbreviation{SC}			{Sequentielle Konsistenz (sequential consistency)}
\abbreviation{WC}			{Schwache Konsistenz (weak consistency)}
\abbreviation{RC}			{Freigabekonsistenz (release consistency)}
			% Glossar   
    \chapter{Erkl�rung der Kandidatin / des Kandidaten}

\begin{description}[$\Box$~]
\item[$\Box$] Die Arbeit habe ich selbstst�ndig verfasst und keine anderen als die angegebenen Quellen und Hilfsmittel verwendet.\\

\item[$\Box$] Die Arbeit wurde als Gruppenarbeit angefertigt. Meine eigene Leistung ist\\
...\\

Diesen Teil habe ich selbstst�ndig verfasst und keine anderen als die angegebenen Quellen und Hilfsmittel verwendet. \\

Namen der Mitverfasser: ...

\end{description}

\vspace{2cm}

\begin{minipage}[t]{3cm}
\rule{3cm}{0.5pt}
Datum
\end{minipage}
\hfill
\begin{minipage}[t]{9cm}
\rule{9cm}{0.5pt}
Unterschrift der Kandidatin / des Kandidaten
\end{minipage}	% Selbstst�ndigkeitserkl�rung
\end{appendix}

\end{document}
